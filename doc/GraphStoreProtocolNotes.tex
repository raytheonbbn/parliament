% !TEX encoding = UTF-8 Unicode
% !TEX TS-program = XeLaTeX
% !BIB TS-program = biber

\documentclass[12pt,letterpaper,draft]{article}

\usepackage{parskip}	% Begin paragraphs with empty line, not indent
\usepackage{xcolor}
\usepackage{titling}
\usepackage{fnpct}
\usepackage{widows-and-orphans}
%\usepackage[inline]{enumitem}
\usepackage{acro}
%\usepackage[final]{graphicx}
%\usepackage{booktabs}
%\usepackage{array}
%\usepackage{ragged2e}
%\usepackage{colortbl}
%\usepackage{rotating}
%\usepackage{marginnote}
%\usepackage{relsize}
%\usepackage{fancyhdr}
%\usepackage{lastpage}
%\usepackage{lipsum}
%\usepackage{amsmath}
%\usepackage{amssymb}
%\usepackage[retainorgcmds]{IEEEtrantools}
%\usepackage{skmath}
%\usepackage{tikz}
%\usepackage{path}
%\usepackage{dirtree}
%\usepackage[firstpage]{draftwatermark}
%\usepackage{todonotes}
%\usepackage[T1]{fontenc}		% Enables curly braces in emails within \institute
\usepackage[style=numeric,backend=biber]{biblatex}
\usepackage{fontspec,xltxtra,xunicode}
%\usepackage{iffont}
%\usepackage{unicode-math}
\usepackage[final]{hyperref}	% Should be the last use-package command

%\newcommand{\red}[1]{{\color{red}#1}}
%\newcommand{\paraleader}[1]{\textbf{#1}}
%\newcommand{\urlcite}[1]{\footnote{\url{#1}}}
%\newcommand{\headingrule}{\midrule[\heavyrulewidth]}
%\newcommand{\classmark}{\textbf{\emph{UNCLASSIFIED}}}
%\newcommand{\themestatement}[1]{\framebox{\parbox{0.97\columnwidth}{\emph{#1}}}}

% !TEX encoding = UTF-8 Unicode
% !TEX TS-program = XeLaTeX

\DeclareAcronym{bbn}{
	short					=BBN,
	long					=RTX BBN Technologies,
	short-plural		='s,
	long-plural			=',
}
\DeclareAcronym{cdt}{
	short					=CDT,
	long					=Eclipse C/C++ Developers Toolkit,
	extra					={A set of Eclipse plug-ins to facilitate C and C++ development in that environment.},
}
\DeclareAcronym{daml}{
	short					=DAML,
	long					=DARPA Agent Markup Language,
}
\DeclareAcronym{darpa}{
	short					=DARPA,
	long					=Defense Advanced Research Projects Agency,
	short-plural		='s,
	long-plural			='s,
}
\DeclareAcronym{dll}{
	short					=DLL,
	long					=Dynamic Link Library,
	extra					={Often used synonymously with ``shared library'' or ``shared object.''},
}
\DeclareAcronym{http}{
	short					=HTTP,
	long					=HyperText Transfer Protocol,
}
\DeclareAcronym{ide}{
	short					=IDE,
	long					=Integrated Development Environment,
}
\DeclareAcronym{jni}{
	short					=JNI,
	long					=Java Native Interface,
	extra					={A facility within the Java platform to allow Java code to directly call native code.},
}
\DeclareAcronym{jvm}{
	short					=JVM,
	long					=Java Virtual Machine,
}
\DeclareAcronym{kb}{
	short					=KB,
	long					=Knowledge Base,
}
\DeclareAcronym{lmdb}{
	short					=LMDB,
	long					=Lightning Memory-Mapped Database Manager,
}
\DeclareAcronym{nas}{
	short					=NAS,
	long					=Network-Attached Storage,
}
\DeclareAcronym{owl}{
	short					=OWL,
	long					=Web Ontology Language,
	short-indefinite	=an,
	long-indefinite	=a,
}
\DeclareAcronym{pmnt}{
	short					=Parliament,
	long					=Parliament™,
	short-plural		='s,
	long-plural			='s,
	extra					={BBN's triple store, so named because ``parliament'' is the collective noun for a group of owls.  A triple store is a specialized database tuned to the unique needs of the Semantic Web data representation.},
	pdfstring			=Parliament/'s,
}
\DeclareAcronym{raid}{
	short					=RAID,
	long					=Redundant Array of Inexpensive Disks,
}
\DeclareAcronym{rdf}{
	short					=RDF,
	long					=Resource Description Framework,
}
\DeclareAcronym{rdfs}{
	short					=RDFS,
	long					=RDF Schema,
}
\DeclareAcronym{san}{
	short					=SAN,
	long					=Storage Area Network,
}
\DeclareAcronym{sparql}{
	short					=SPARQL,
	long					=SPARQL Protocol and RDF Query Language,
	extra					={If this seems confusing, it is because SPARQL is a recursive acronym.},
}
\DeclareAcronym{swrl}{
	short					=SWRL,
	long					=Semantic Web Rule Language,
}
\DeclareAcronym{url}{
	short					=URL,
	long					=Uniform Resource Locator,
	extra					={The address of a page on the World Wide Web, typically starting with the prefix ``http:'' or ``https:''.},
}
\DeclareAcronym{w3c}{
	short					=W3C,
	long					=World Wide Web Consortium,
}

\bibliography{UserGuide/references}

\title{\acl{pmnt} Graph Store Protocol Implementation Notes}
\author{Ian Emmons}
\date{\today}

\defaultfontfeatures{Mapping=tex-text}
\setmainfont{Georgia}
\setsansfont{Verdana}[Scale=MatchLowercase]
\setmonofont{Menlo}[Scale=MatchLowercase,PunctuationSpace=WordSpace,Mapping=]
%\setmonofont{Consolas}[Scale=MatchLowercase,PunctuationSpace=WordSpace,Mapping=]
%\setmonofont{Andale Mono}[Scale=MatchLowercase,PunctuationSpace=WordSpace,Mapping=]
%\setmathfont{TeX Gyre Bonum Math}
%\setmathfont{TeX Gyre Pagella Math}
%\setmathfont{STIX Math}
%\setmathfont{latinmodern-math.otf}

%\urlstyle{rm}
\definecolor{hyperlinkcolor}{rgb}{0,0,0.625}
\hypersetup{
	final=true,
	unicode=true,
	bookmarksnumbered=true,
	pdftitle={\thetitle},
	pdfauthor={\theauthor},
	colorlinks=true,
	allcolors=hyperlinkcolor
}

\hyphenation{Arc-GIS Arc-Map}
\hyphenation{co-ref-er-ence co-ref-er-enc-es}
\hyphenation{da-ta-base da-ta-bas-es}
\hyphenation{geo-spa-tial}
\hyphenation{im-mu-ta-ble}
\hyphenation{ra-pid-ly}
\hyphenation{ray-the-on}
\hyphenation{serv-let serv-lets}
\hyphenation{trac-ta-ble}
\hyphenation{war-fight-er war-fight-ers}

\begin{document}
\acuse{pmnt} % Only long & short forms, never "first appearance"
\maketitle

%%%%%%%%%%%%%%%%%%%%%%%%%%%%%%%%%%%%%%%%%%%%%%%%%%%%%%%%
\section{Introduction}
\label{sec:intro}

\autocite{SparqlGraphStoreProtocol}



%%%%%%%%%%%%%%%%%%%%%%%%%%%%%%%%%%%%%%%%%%%%%%%%%%%%%%%%
\section{Limitations and Caveats}
\label{sec:limitations}

\ac{pmnt} implements only Indirect Graph Identification



%%%%%%%%%%%%%%%%%%%%%%%%%%%%%%%%%%%%%%%%%%%%%%%%%%%%%%%%
\section{To-Do List}
\label{sec:todo}

Graph names MUST be absolute IRIs and the server MUST respond with a 400 Bad Request if not

If the Accept header is not provided with a GET request, the server MUST return one of RDF XML, Turtle, or N-Triples.

For operations involving an RDF payload (PUT and POST):
\begin{itemize}
	\item The server MUST parse the RDF payload according to media type specified in the Content-Type header if it is provided in the request.

	\item If the header is not provided, and the implementation has a routine that can guess the type by the content of the resource or via the extension of the file it was loaded from, and such a routine reported that the resource was clearly some other document format and not RDF/XML, then the implementation MAY attempt to parse the document using this format.

	\item Otherwise, if this header is not provided, the server SHOULD attempt to parse the RDF payload as RDF/XML.
\end{itemize}

in response to operations involving an RDF payload, if the attempt to parse the RDF payload according to the provided Content-Type fails then the server MUST respond with a 400 Bad Request.

A request using an unsupported HTTP verb in conjunction with a malformed or unsupported request syntax MUST receive a response with a 405 Method Not Allowed.

If the RDF graph content identified in the request does not exist in the server, and the operation requires that it does, a 404 Not Found response code MUST be provided in the response.

If a clients issues a POST or PUT with a content type that is not understood by the graph store, the implementation MUST respond with 415 Unsupported Media Type.

GET: In the event that the specified representation format is not supported, a 406 Not Acceptable response code SHOULD be returned.

POST: If the request IRI identifies the underlying Graph Store, the origin server MUST create a new RDF graph comprised of the statements in the RDF payload and return a designated graph IRI associated with the new graph. The new graph IRI should be specified in the Location HTTP header along with a 201 Created code and be different from the request IRI.




\acsetup{
	list/name=Glossary,
	list/heading=section}
\printacronyms

\sloppy
\printbibliography[heading=bibnumbered]

\end{document}
